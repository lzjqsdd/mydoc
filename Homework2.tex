\documentclass[]{article}

%opening
\title{Assignment 2}
\author{lvzejun 201528013229115}

\usepackage{fancyhdr}

\usepackage{amsmath}
\usepackage{amssymb}
\usepackage{indentfirst}
\usepackage{wasysym}

\usepackage{listings}
\usepackage{xcolor}
\usepackage{color}

\usepackage{caption}
\usepackage{algorithm}
\usepackage{algpseudocode}

\usepackage{fontspec}
\usepackage{float}

\pagestyle{fancy}
\lhead{\fontfamily{Monaco} \bfseries Dynamic Programming}
\chead{}
\rhead{\fontfamily{Monaco} \bfseries lvzejun 201528013229115} 

\setmonofont[Mapping={}]{Monaco}	%英文引号之类的正常显示,相当于设置英文字体
\setsansfont{Monaco} %设置英文字体 Monaco, Consolas,  Fantasque Sans Mono
%\setmainfont{Monaco} %设置英文字体

\definecolor{mygreen}{rgb}{0,0.6,0}
\definecolor{mygray}{rgb}{0.5,0.5,0.5}
\definecolor{mymauve}{rgb}{0.58,0,0.82}

\lstset{ %
  backgroundcolor=\color{white},   % choose the background color; you must add \usepackage{color} or \usepackage{xcolor}
  basicstyle=\footnotesize\ttfamily,        % the size of the fonts that are used for the code
  breakatwhitespace=false,         % sets if automatic breaks should only happen at whitespace
  breaklines=true,                 % sets automatic line breaking
  captionpos=b,                    % sets the caption-position to bottom
  commentstyle=\color{mygreen},    % comment style
  deletekeywords={...},            % if you want to delete keywords from the given language
  escapeinside={\%*}{*)},          % if you want to add LaTeX within your code
  extendedchars=true,              % lets you use non-ASCII characters; for 8-bits encodings only, does not work with UTF-8
%  frame=single,	                   % adds a frame around the code
  keepspaces=true,                 % keeps spaces in text, useful for keeping indentation of code (possibly needs columns=flexible)
  keywordstyle=\color{blue}\bf,       % keyword style
  language=C++,                 % the language of the code
  otherkeywords={*,...},            % if you want to add more keywords to the set
  numbers=left,                    % where to put the line-numbers; possible values are (none, left, right)
  numbersep=10pt,                   % how far the line-numbers are from the code
  numberstyle=\tiny\color{mygray}, % the style that is used for the line-numbers
  rulecolor=\color{black},         % if not set, the frame-color may be changed on line-breaks within not-black text (e.g. comments (green here))
  showspaces=false,                % show spaces everywhere adding particular underscores; it overrides 'showstringspaces'
  showstringspaces=false,          % underline spaces within strings only
  showtabs=false,                  % show tabs within strings adding particular underscores
  stepnumber=1,                    % the step between two line-numbers. If it's 1, each line will be numbered
  stringstyle=\color{mymauve},     % string literal style
  tabsize=2,	                   % sets default tabsize to 2 spaces
  title=\lstname                   % show the filename of files included with \lstinputlisting; also try caption instead of title
}

\begin{document}


\maketitle
\tableofcontents
\newpage
%=================================================================================================== %

\section{Money robbing}
A robber planning to rob houses along a street. Each house has a certain
amount of money stashed, the only constraint stopping you from robbing
each of them is that adjacent houses have security system connected and
it will automatically contact the police if two adjacent houses were broken
into on the same night.\\

1.Given a list of non-negative integers representing the amount of money
of each house, determine the maximum amount of money you can rob
tonight without alerting the police.\\\\
2. What if all houses are arranged in a circle?
\subsection{Optimal subproblems And DP equation}
\subsubsection{Question1}
Denote the optimal solution value as rob\_max(i) which means the max money can be robbed when consider the $i^{th}$ house,and m stand for the count of money of each house.Suppose we start the house index from 0,then:\\
(1)If there exists only 1 house,rob\_max(0) = $m_0$;\\
(2)If there exists 2 houses,chose the max one.max\{$m_0$,$m_1$\}.\\
(3)If there exists more than 2 houses,we can summarizing two cases:\\

$$
\begin{cases}
rob\_max(i-2)+ m_i&  \text{if chose to rob the $i^{th}$ house} \\
rob\_max(i-1) &
\end{cases}
$$

Optimal sub-structure property:\\
$$
rob\_max(i) = \begin{cases}
m_0 &  \text{n=1} \\
max\{m_0,m_1\} &  \text{n=2} \\
max\{rob\_max(i-2)+m_i,rob\_max(i-1)\}& \text{$n\geqslant2$} 
\end{cases}
$$

\subsubsection{Question2-All houses in a circle}
If all houses are arrange in a circle,it means that the robber can't rob the first and last house in the sametime,either the first one or the last one.So we can divide the problem into two subquestions,(1)find max of 1 to last-1,(2)find max of 2-last,then find max.
\subsection{Proof}
%We define rob\_max(i) to be max money.Suppose there exists another rob_max'(i) 
The algoritm evaluates the recurrence rob\_max(i),Note that since we evaluate rob\_max(i) as i increases from 1 to n,all values for subproblems referenced by the recurrence for rob\_max(i) have already been computed.At the end,our algorithmlreturns the maximum value in the rmax array.
\subsection{Code}

\begin{algorithm}[H]
\caption{Money robbing} \label{Dynamic Programming}
\begin{lstlisting}[]

#include <iostream>
#include <cstdio>
#include <cstdlib>

using namespace std;

int max(int a,int b)
{
	return a>b ? a:b;
}

int rob_max(int a[],int n) //n stand for the size of array
{
	int rmax[n];
	if(n==1) return a[0];
	else if(n==2) return max(a[0],a[1]);
	else
	{
		rmax[0] = a[0];
		rmax[1] = max(a[0],a[1]);
		for(int i=2;i<n;i++)
		{
			rmax[i] = max(rmax[i-2]+a[i],rmax[i-1]);
		}
	}
	return rmax[n-1];
}

int main()
{
	//test cases;
	int a[5] = {1,3,4,2,1};
	int b[4] = {50,1,1,50};
	int c[8] = {10,13,23,17,6,11,18,16};
	cout<<rob_max(a,5)<<endl;
	cout<<rob_max(b,4)<<endl;
	cout<<rob_max(c,8)<<endl;

	//If all houses are in a circle
	cout<<max(rob_max(&c[1],7),rob_max(c,7))<<endl;
	return 0;
}
\end{lstlisting}
\end{algorithm}
\subsection{Complexity of Algorithm}
Analysis function rob\_max,there exist only one for loop,from 2 to n,so time complexity is O(n).And we use one array rmax[n] to store the optimal result for every step,
so the space complexity is O(n)
%=================================================================================================== %

\section{Minimum path sum}
Find the minimum path sum of a triangle from top to bottom. Each step you may move to adjacent numbers on the row below.
For example, given the following triangle:

\centerline{2}
\centerline{3 4}
\centerline{6 5 7}
\centerline{4 1 8 3}

The minimum path sum from top to bottom is 11 ( i.e., 2 + 3 + 5 + 1 = 11).
\subsection{Optimal subproblems And DP equation}
If we use two-dimensional array a[n][n] to store the input triangl,and layer i will have i nodes.Denote the optimal solution value as minsum[i][j],which means the minimum sum when reach the current node.The node a[i][j] can come from a[i-1][j-1] and a[i-1][j],and a[i][0] can only from a[i-1][0],a[i][i-1] can only come from a[i-1][i-1].\\
So,we can define the optimal sub-structure property:\\
	$$
	minpath[i][j] = \begin{cases}
	minsum[i-1][j]+a[i][j] & \text{j=0} \\
	minsum[i-1][j-1]+a[i][j] & \text{j=i} \\
	min(minsum[i-1][j-1],minsum[i-1][j])+a[i][j] & \text{other}
	\end{cases}
	$$

\subsection{Proof}
The algoritm evaluates the recurrence minsum[i][j],Note that since we evaluate minsum[i][j] as i increases from 1 to n and j increase from i to n,all values for subproblems referenced by the recurrence for minsum[i][j] have already been computed.And every time we chose the minimum one.At the end,our algorithmlreturns the minimum value in the minsum array.Suppose there exists another minsum'[i][j],there must be more mininum minsum[i-1][j-1] or minsum[i-1][j],which confict with our define.
\subsection{Code}
\begin{algorithm}[H]
\caption{Minimum path sum}
\begin{lstlisting}
#include <iostream>
#include <cstdio>
#include <cstdlib>
#include <climits>
using namespace std;

int a[4][4];
int minsum[4][4];
int n;

int min(int a,int b)
{
	return a<b ? a:b;
}

int minpath()
{
	int m=INT_MAX; 
	minsum[0][0]=a[0][0];
	for(int i=1;i<n;i++)
		for(int j=0;j<=i;j++)
		{
			if(j==0) minsum[i][j] = minsum[i-1][j]+a[i][j];
			else if(j==i) minsum[i][j] = minsum[i-1][j-1]+a[i][j];
			else
				minsum[i][j] = min(minsum[i-1][j],minsum[i-1][j-1]) + a[i][j];
		}
	for(int k=0;k<n;k++)
		m = min(m,minsum[n-1][k]);
	return m;
}

int main()
{
	//test case
	n=4;
	a[0][0]=2;
	a[1][0]=3;a[1][1]=4;
	a[2][0]=6;a[2][1]=5;a[2][2]=7;
	a[3][0]=4;a[3][1]=1;a[3][2]=8;a[3][3]=3;

	cout<<minpath()<<endl;
}
\end{lstlisting}
\end{algorithm}
\subsection{Complexity of Algorithm}
Analysis function minpath(),we travelsal every node once.So the time complexity is O($n^2$).we use two-dimensional array to store every optimal sub-problems.So the
space complexity is O($n^2$)

%=================================================================================================== %

\section{Decoding}
A message containing letters from A-Z is being encoded to numbers using the following mapping:\\
\centerline{A: 1}
\centerline{B: 2}
\centerline{...}
\centerline{Z: 26}\\\\
Given an encoded message containing digits, determine the total number
of ways to decode it.
For example, given encoded message “12”, it could be decoded as “AB”
(1 2) or “L” (12). The number of ways decoding “12” is 2.
\subsection{Optimal subproblems And DP equation}
By Observing the encode,we can find that,when decoding,Only two neighboring characters can combian.And character '0' can't be divided alone.So we should judge whether the character can be divide alone,whether can combian with front character.Denote the optimal solution value as count[n] which means the most solutions of decoding.\\
So,we can define the optimal sub-structure property:\\
	$$
	count[i] = \begin{cases}
	count[i-1] & \text{can divide} \\
	0 & \text{can't divide} \\
	count[i]+count[i-2] & \text{can combian} \\
	\end{cases}
	$$
\subsection{Proof}

The algoritm evaluates the recurrence count[i],Note that since we evaluate count[i] as i increases from 1 to n,all values for subproblems referenced by the recurrence for count[i] have already been computed.At the end,our algorithmlreturns the count of solutions in the count[n] array.

\subsection{Code}
\begin{algorithm}[H]
\caption{Minium Path Sum}
\begin{lstlisting}
#include <iostream>
#include <cstdio>
#include <cstdlib>
#include <cstring>
using namespace std;

bool cancom(char a,char b)
{
	if(a=='1') return true;
	else if(a=='2' && b<='6') return true;
	else
		return false;
}
int numdecode(char *s)
{
	int len = strlen(s);
	int count[len+1];

	count[0]=1;
	count[1]=1;

	if(len==0) return 0;
	if(s[0]=='0') return 0;
	if(len==1) return 1;
	for(int i=2;i<=len;i++)
	{
		//cannot divide
		if(s[i-1]=='0')
			count[i]=0;
		else
			count[i]=count[i-1];
		if(cancom(s[i-2],s[i-1]))
		{	
			cout<<"here"<<endl;
			count[i]+=count[i-2];
		}
		if(count[i]==0) return 0;
	}
	return count[len];
}

int main()
{
	char s[100];
	while(~scanf("%s",s))
	{
		cout<<numdecode(s)<<endl;
	}
	return 0;
}

\end{lstlisting}
\end{algorithm}
\subsection{Complexity of Algorithm}
Analysis function numdecode(),we only use one for loop tocalculate the solutions of decode,so the time complexity is O(n).And we use one-dimensional array to
store the optimal sub-problem solutions,so the space complexity is O(n).
%=================================================================================================== %

\section{Maximum profit of transactions}
Say you have an array for which the i-th element is the price of a given stock on day i.
Design an algorithm and implement it to find the maximum profit. You may complete at most two transactions.
Note: You may not engage in multiple transactions at the same time (ie,you must sell the stock before you buy again).

\subsection{Optimal subproblems And DP equation}
For this problem,we can consider more simple problem.Suppose we can complete only one transaction.we denote p1[n] to store maximum profit,define mmin to maintain the minimum stock for every day i.max1 maintain the max profit for current day.Define stock[n] for input.

So,we can define the optimal sub-structure property:\\

\centerline{p1[i] = max(max1,stock[i]-mmin)}

Now talk about this problem,we can complete at most two transactions.If we complete two
transactions,the max profit must come from day 1 to k and  day k+1 to n.So we need to calculate max profit from day k to n(k can be 1 to n).It's similar to calculate  max profit for day 1 to n,just from day n to 1 to calcute.We use p2[n] to store it.P[i] means the max profit for day i to n.And we calculate the max vaule of p1[k] and p2[k+1].

\subsection{Code}
\begin{algorithm}[H]
\caption{Find Maxinum Profit}
\begin{lstlisting}
int maxprofit(int* p, int size) {

   int max1 = 0; int mmin=INT_MAX; int p1[size];
	for(int i=0;i<size;i++)
	{
		if(p[i] < mmin) mmin = p[i];
		int c = p[i]-mmin;
		if(c>max1) max1=c;
		p1[i]=max1;
	}

	int max2=0;	int mmax=INT_MIN; int p2[size];
	for(int i=size-1;i>=0;i--)
	{
		if(p[i] > mmax) mmax=p[i];
		int c=mmax-p[i];
		if(c>max2) max2=c;
		p2[i]=max2;
	}

   int max_result = 0;
   for(int k=0;k<size-2;k++)
   {
	   int sum = p1[k]+p2[k+1];
	   if(sum>max_result)
		   max_result= sum;
   }
   return max_result > max1 ? max_result: max1;
}

int main()
{
	int a[4]={1,5,1,5,6};
	cout<<maxprofit(a,5)<<endl;
}

\end{lstlisting}
\end{algorithm}
\subsection{Complexity of Algorithm}
Analysis function maxprofit(),we use three for loop independent.So time complexity is O(n).And use p1 and p2 to store the max profit,the space complexit is O(n).
\end{document}

